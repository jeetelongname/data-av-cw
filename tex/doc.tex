\documentclass{report}
\usepackage[colorlinks=true,linkcolor=black,urlcolor=blue,citecolor=cyan]{hyperref} %% links
\usepackage[style=apa, backend=biber]{biblatex}
\usepackage{csquotes}
\usepackage[british]{babel}
\addbibresource{data-av.bib}
\usepackage{graphicx}
\usepackage[top=1.5cm]{geometry}
\title{Some Statistics of Gun Violence In the United States between the years of
2015 to 2018}

\author{Jeetadtya Chatterjee --- UP2063130}
\begin{document}
\maketitle
\tableofcontents
\chapter{The Data}\label{chapt:data}
To curate the data I spent time coercing the data into types I can work with. I
discarded all rows before the first of January 2015 as specified in the
document. When doing numerical calculations I treated nil values as a neutral
element. This adds a level of error to each column but allows for the whole set
of columns to pull as much information from the data set.
Other than that I have had to scale a lot of my calculations to work on an
average of months due to 2018 only contains the first 3 month of
data. This allows us to extrapolate what the entirety of 2018 would have looked
like but as shown in~\ref{chapt:plot4} there seems to be a dip in cases in the
first three months of the year showing a kind of shrinking overall even when
this may not be the case.

\chapter{The average monthly, Incidents, Injuries and Deaths for each year.}\label{chapt:plot1}
For my first visualisation. I have calculated the average monthly incidents for
each year. This is showing a slight increase every year with a drop off in 2018.
see~\ref{chapt:data} for the reason why. This allows us to get a feel for
gun violence on a national scale but does not tell us if this is homogeneous
over the USA or if there are a couple of hot spots that are accounting for the
overall trend.

\begin{figure}[htbp]
	\centerline{\includegraphics[scale = 0.6]
		{/home/jeet/code/R/data-av/coursework/scripts/plot_1/plot1.png}}
	\caption{\label{fig:1} A bar chart showing the average monthly incidents, injuries and deaths}
\end{figure}

\chapter{The average monthly incidents, injuries and deaths for each year; broken down by state}\label{chapt:plot2}
\begin{quote}
	\textbf{Link to the visualisation:
		\url{https://jeetelongname.shinyapps.io/plot_2/}}
\end{quote}
For this visulisation I have used population estimates from the US Census bureau
\autocite{bureau2019NationalState} to calculate the incident rate per 100,000
people.

Here we can see how that District of Columbia seems to be much higher than the
average. It as a state seems to lead the numbers of incidents per 100,000. With
a large spike in 2017, going from a mean value of incidents of 6, to 13. The
District of Columbia is the capital of the United States and is just a single
City. There is no other state that can compare to this. But it suggests a
correlation of Population density to gun incidents.

\chapter{Scatter map of gun incidents each year}\label{chapt:plot3}
\begin{quote}
	\textbf{Link to the visualisation:
		\url{https://jeetelongname.shinyapps.io/plot_3/}}
\end{quote}
For this visulsation I use the Simple features package
\autocite{pebesmaSimpleFeaturesStandardized2018} which provides support for
spatial data. For the actual spatial data of US states I used
\autocite{CRANPackageUSA}.

Here we can see that how a lot of gun incidents are congregating around certain
areas. We call these areas towns and cities. These are most pronounced in the
states of Nevada, New York and Alaska. Going back to our special outlier in the
District of Columbia we can see a clear divide where the North Western portion
of the city is mostly free from these incidents while a majority of the center
and the south bank has many more incidents.

\chapter{Time series plot of incidents in each state over each year }\label{chapt:plot4}
\begin{quote}
	\textbf{Link to the visualisation:
		\url{https://jeetelongname.shinyapps.io/plot_4/}}
\end{quote}

We have discussed where these incidents have been taking place but not when nor
how. In this next visualisation we plot each incident as a function of time.
Here we see that incidents in a lot of states seem to go down before rising back
up during the summer months. This is most pronounced in the Northern states with
the outlier of Florida which sees a rise approaching June with a sizable dip as
we approach the winter months. This pattern is seen clearly on a national
level.

\chapter{Propotional stacked bars showing age demographics over the time period}\label{chapt:plot5}
For my final visualisation I want to show the age group mostly affected by gun
violence. This visualisation does not separate suspects from victims, it just
shows who is taking part and being affected by gun incidents. The proportions
are the mean number of people taking part in each year.

Here we can see how Adults (Over 18s) are the most susceptible to gun
incidents. With teens (12 to 17) being the next big category and then children
(0 to 11) being the smallest. We see no significant rise
in the proportions with the slight rise in 2018 coming down to the incomplete
nature of the data as mentioned in~\ref{chapt:data}

\begin{figure}[htbp]
	\centerline{\includegraphics[scale = 0.6]{/home/jeet/code/R/data-av/coursework/scripts/plot_5/plot5.png}}
	\caption[]{\label{fig:plot5} A proportional bar chart showing the relative
		change in the age groups affected by gun violence}
\end{figure}


\printbibliography{}
\end{document}
